\section{Verkettete Listen}

\begin{frame}[fragile]
\frametitle{Linked List}
Warum Listen?
\begin{itemize}
    \item erste Node wird \textit{head} genannt (letzte wird \textit{tail} genannt)
    \item jede Node hat einen Zeiger auf nächste Node
    \item sie sind dynamisch vergrößerbar
    \item effizient, wenn wir einfügen oder löschen wollen
\end{itemize}
TODO: Grafik für Listen
Welche Listen kennen wir?
\end{frame}

\begin{frame}[fragile]
\frametitle{Beispielaufgabe}
Schreiben Sie eine Klasse \textit{Test} die eine Liste von \textit{Exercise} Objekten enthält
\begin{enumerate}
    \item ein \textit{Exercise} Objekt speichert einen Namen und eine Aufgabenstellung
    \item Ausgabe eines \textit{Exercise} Objektes
    \begin{lstlisting}
        <Name der Aufgabe>:
        <Aufgabenstellung>
    \end{lstlisting}
    \item implementieren sie eine Methode \textit{show}, die das erste  \textit{Exercise} Objekt der Liste ausgibt
    \item implementieren sie eine Methode \textit{rotate}, die zum nächsten \textit{Exercise} Objekt wechselt, dieses Ausgibt und das vorherige hinten an die Liste anhängt
    \item implementieren sie eine Methode \textit{showAll}, die alle \textit{Exercise} Objekte ausgibt 
\end{enumerate}
\end{frame}

\begin{frame}[fragile]
\frametitle{Mini-Game}
    Entwickeln sie ihr Game weiter:
    \begin{enumerate}
        \item überlegen Sie sich sinnvolle Interfaces (\textit{useable}, \textit{weapon}, etc.)
        \item implementieren Sie eine Methode \textit{sort}, die verschiedene Arten für das Sortieren des Inventares bereitstellt (bspw. alphabetisch, nach gewicht, etc.)
        \item erstellen sie eine Spieler Klasse
        \item das Spiel verwaltet eine Liste von Spieler Objekten
    \end{enumerate}
\end{frame}