

\definecolor{codegreen}{rgb}{0,0.6,0}
\definecolor{codegray}{rgb}{0.5,0.5,0.5}
\definecolor{codepurple}{rgb}{0.58,0,0.82}
\definecolor{backcolour}{rgb}{0.95,0.95,0.92}
\definecolor{numbercolour}{rgb}{0.95,0.95,0.02}

\lstdefinestyle{mystyle}{
    language=java,
    basicstyle=\ttfamily\tiny,
    breakatwhitespace=false,         
    breaklines=true,                 
    captionpos=b,                    
    keepspaces=true,                 
    numbers=left,                    
    numbersep=5pt,                  
    showspaces=false,                
    showstringspaces=false,
    showtabs=false,                  
    tabsize=4
}

\lstset{style=mystyle}

\section{Bäume}

\begin{frame}[fragile]
\frametitle{Bäume}
\begin{center}
    Baum:
\end{center}

\end{frame}

\begin{frame}[fragile]
\frametitle{Bäume}
\begin{center}
    Baum:\\
   \end{center} 
        \begin{minipage}{0.5\textwidth}
             \begin{itemize}
        \item Jedes Element (Knoten): linker und rechter Nachfolger (Kind) (= „Binärbaum“)

  
    \end{itemize}
	    \end{minipage}

	  \hfill
	  \begin{minipage}{0.5\textwidth} \vspace{-1.5cm}  \hspace{1cm}
		 \includegraphics[height=0.8\textheight]{images/Baum_Grundstruktur-01.png}
	  \end{minipage}
    
      \note{
        XXXXXXXXXXXXX
      }

\end{frame}



\begin{frame}[fragile]
\frametitle{Bäume}
\begin{center}
    Baum:\\
   \end{center} 
        \begin{minipage}{0.5\textwidth}
             \begin{itemize}
     
\item Ordnung: 
    \begin{itemize}
        \item Alle Knoten im linken Teilbaum kleiner als Knotenwert
        \item alle Knoten im rechten Teilbaum größer
    \end{itemize}
(= „binärer Suchbaum“)
\item  keine doppelten Werte
    \end{itemize}
	    \end{minipage}

	  \hfill
	  \begin{minipage}{0.5\textwidth} \vspace{-3.5cm}  \hspace{1cm}
		 \includegraphics[height=0.8\textheight]{images/Baum_Grundstruktur-01.png}
	  \end{minipage}
    
      \note{
        XXXXXXXXXXXXX
      }

\end{frame}

\begin{frame}[fragile]
\frametitle{Bäume}
\begin{center}
    Baum:\\
   \end{center} 
        \begin{minipage}{0.5\textwidth}
             \begin{itemize}
        \item 2 ist linkes Kind von 3.
\item 3 ist Elternknoten von 2 und 4.
\item 3 ist die Wurzel des linken
Teilbaums von 5.
\item Der Baum hat die Tiefe 3.
    \end{itemize}
	    \end{minipage}

	  \hfill
	  \begin{minipage}{0.5\textwidth} \vspace{-1.5cm}  \hspace{1cm}
		 \includegraphics[height=0.45\textheight]{images/Baum_Grundstruktur-02.png}
	  \end{minipage}
    
      \note{
        XXXXXXXXXXXXX
      }

\end{frame}

\begin{frame}[fragile]
\frametitle{Bäume - Umsetzung}
\begin{lstlisting}[language=java]
public class BinarySearchTree {

    private class BinaryNode {
    private int element;
    private BinaryNode left, right;


    private BinaryNode(int element) {
        this.element = element;
    }
}

private BinaryNode root;

\end{lstlisting}

\end{frame}

\begin{frame}[fragile]
\frametitle{Bäume - Einfügen}
\begin{lstlisting}
public void insert(int newNumber) {
    // Sonderfall: leerer Baum
    if (root == null) {
        root = new BinaryNode(newNumber);
        return;
    }

    BinaryNode parent = null;
    BinaryNode child = root;
    
    while (child != null) {
        parent = child;
        if (newNumber == child.element) {
            // Zahl bereits im Baum vorhanden
            return;
        } else if (newNumber < child.element) {
            child = child.left;
        } else {
            child = child.right;
        }
    }
    if (newNumber < parent.element) {
        parent.left = new BinaryNode(newNumber);
    } else {
        parent.right = new BinaryNode(newNumber);
    }
}

\end{lstlisting}

\end{frame}


\begin{frame}[fragile]
\frametitle{Bäume - Minimum finden}
\begin{lstlisting}
public int minimum() {
    if(root == null) {
        throw new java.util.NoSuchElementException();
    }

    BinaryNode current = root;
    while(current.left != null) {
        current = current.left;
    }

    return current.element;
}
\end{lstlisting}

\end{frame}



\begin{frame}[fragile]
\frametitle{Einschub: Exceptions}
\begin{lstlisting}
    throw new [Exception]
\end{lstlisting}
Throwing (werfen) $\xrightarrow{}$ sorgt für das auslösen einer Exception. 
\begin{lstlisting}
    Try{
        //Was versucht werden soll
    } catch([Exceptiontyp] e) {
        // Code zur Fehlerbehandlung
    }
\end{lstlisting}
\begin{itemize}
    \item Fängt Exceptions ab
    \item verhindert Programmabsturz.
    \item Errors, können auch abgefangen werden! Ist aber fast immer eine schlechte Lösung!
\end{itemize}
    \textbf{Vorsicht!} Alle Exceptions folgen einer Umfangreichen Vererbungshierarchie. Wenn ein zu hoher Obertyp gewählt wird, kann die tatsächliche Exception verdeckt werden oder kann nicht abgefangen werden

\end{frame}

\begin{frame}[fragile]
\frametitle{Einschub: Exceptions}
\begin{center}
\includegraphics[height=0.8\textheight]{images/ExceptionClassHierarchy.png}
\tiny $https://www.protechtraining.com/static/bookshelf/java\_fundamentals\_tutorial/images/ExceptionClassHierarchy.png$

\end{center}
\end{frame}
