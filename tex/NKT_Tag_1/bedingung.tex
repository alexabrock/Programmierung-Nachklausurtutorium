\section{Kontrollstrukturen}
\subsection{Verzweigungen}
\begin{frame}[fragile]
\frametitle{Verzweigung}
\begin{lstlisting}
    if(x){                  //kann aleine genutzt werden
        //do X              
    } else if(y) {          //falls spezieller Fall y
        //do Y
    } else {                //default Fall
        //do Z
    }
\end{lstlisting}
\begin{itemize}
    \item falls viele \textit{else if's} auftreten sollten wir eher ein \textit{switch statement} verwenden
\end{itemize}

\end{frame}

\begin{frame}[fragile]
\frametitle{Schleifen - While}
\begin{lstlisting}
    //while Schleife
    while(x){                  //solange die Bedingung x zutrifft
        //do X              
    } 
    //do-while Schleife
    do{                        
    } while(x)
\end{lstlisting}
\begin{itemize}
    \item Achtung vor Endlosschleifen!
\end{itemize}
\end{frame}

\begin{frame}[fragile]
\frametitle{Schleifen - \textit{for}}
\begin{lstlisting}
    for(int i = 0; i < 5; i++){
        //do X
    }
\end{lstlisting}
\begin{itemize}
    \item häufig benutzt, um Collections durchzugehen
    \item \textit{for-each} Schleife:
\end{itemize}
\begin{lstlisting}
    for (int itVar : array){
        //do X
    }
\end{lstlisting}
\begin{itemize}
    \item hierbei bitte aufpassen. Mit der \textit{for-each} Schleife kann nicht der Inhalt des Arrays geändert werden. Dies hat Performance Gründe
\end{itemize}
\end{frame}

\subsection{Logische Operatoren}
\begin{frame}[fragile]
\frametitle{Logische Operatoren}
Logische Operatoren werden in Schleifenköpfen benötigt. Die Bedingung dafür, wie lange die Schleife läuft, ist also ein komplexer logischer Operator:
\begin{itemize}
    \item \texttt{||}
    \item \texttt{\&\&}
    \item \texttt{==}
    \item \texttt{<} oder \texttt{>}
    \item \texttt{<=} oder \texttt{>=}
    \item \texttt{|}
    \item \texttt{\&}
\end{itemize}
\begin{lstlisting}
    if(v + 42 <= 69){
        // do X
    }
\end{lstlisting}
\end{frame}

\subsection{Idiomatisch}
\begin{frame}[fragile]
\frametitle{Idiomatisch}
For Schleife $\Leftrightarrow$ While Schleife
\newline
Wenn wir ein Array mit einer \textit{while} Schleife ausgeben wollen geht dies auch: 
\begin{lstlisting}
    int count = 0;
    while(count <= array.length){
        System.out.println(array[count]);
        count++;
    }
\end{lstlisting}
Eine \textit{for} Schleife ist hier jedoch deutlich sinnvoller
\end{frame}