\section{Basics}
\lstset{ % Allgemeine Einstellungen für alle Codeblöcke
    language=C,% Sprache für Syntax-Highlighting
    basicstyle=\ttfamily, % Grundlegender Schriftstil
    keywordstyle=\color{blue},     % Farbe der Schlüsselwörter
    commentstyle=\color{gray},     % Farbe der Kommentare
    stringstyle=\color{red},       % Farbe der Zeichenketten
    numbers=left,                  % Zeilennummern links anzeigen
    numberstyle=\tiny\color{gray}, % Stil der Zeilennummern
    frame=single,                  % Rahmen um den Code
    breaklines=true,               % Automatischer Zeilenumbruch
    captionpos=b, % Bildunterschrift unterhalb des Codes
}

\subsection{Variablendeklaration}

\begin{frame}[fragile] %fragile tag benötigt, da code Listings auftreten
\frametitle{Variablendeklaration}

In Java benutzen wir häufig Variablen, um Daten zu speichern.
Um eine Variable anzulegen benötigen wir einen Datentypen und einen Namen. Das Anlegen einer Variable heißt Deklaration:
\begin{lstlisting}
    int f;
    float g;
    String msg;
\end{lstlisting}
primitive Variablen, die nur deklariert wurden, kriegen einen Standardwert zugewiesen. Bei \textit{int} ist dieser beispielsweise 0
    
\end{frame}

\subsection{Variablen}

\begin{frame}[fragile]
\frametitle{Variableninitialisierung}

Bei der Initialisierung von Variablen weisen wir diesen nun einen Wert zu:
\begin{lstlisting}
    int f = -2;
    float g = 42.69;
    String msg = "Hello World";
\end{lstlisting}

Referenzen gelten hier auch als Werte, die wir zuweisen können:
\begin{lstlisting}
    String importantMessage = msg + "!";
\end{lstlisting}
    
\end{frame}

\subsection{Datentypen}

\begin{frame}{Primitive Datentypen}
Welche primitiven Datentypen gibt es?
\begin{itemize}
\item boolean
\item byte
\item char
\item short
\item int
\item long
\item float
\item double
\end{itemize}
\end{frame}

\begin{frame}{Nicht primitive Datentypen}
\begin{itemize}
    \item Objektreferenzierung
    \begin{itemize}
        \item speichert anstelle des Objektes nur die Referenz auf dem Stack
        \item erlaubt die Einschränkung von Zugriff auf Attribute und Variablen des Objektes
        \item ist häufig nötig, da Stack Platz sehr begrenzt
    \end{itemize}
\end{itemize}

\begin{itemize}
    \item Beispiele
    \begin{itemize}
        \item String
        \item java.awt.Color 
        \item (Arrays)
        \item eigene Klassen
    \end{itemize}
\end{itemize}
\end{frame}

\subsection{String Objekt}

\begin{frame}[fragile]
\frametitle{Die String Klasse}
\begin{itemize}
    \item ein String Objekt enthält eine Zeichenkette in festgelegter Reihenfolge
    \item die Klasse String verfügt über Methoden, die auf dem Objekt selber aufgerufen werden können
\end{itemize}
\begin{lstlisting}
    String msg = "dies ist ein String!";
    String msg2 = msg.toUpperCase();
\end{lstlisting}
\end{frame}
