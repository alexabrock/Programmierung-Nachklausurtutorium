\section{Arrays und Standardeingabe}

\subsection{Arrays}
\begin{frame}[fragile]
\frametitle{Arrays}
Was ist ein Array?
\begin{itemize}
    \item speichert eine genaue Anzahl an Variablen desselben Datentypes
    \item hat eine feste Größe (nicht dynamisch erweiterbar)
    \item nicht initialiserte Felder kriegen einen Standardwert oder null zugewiesen
    \item indices für schnellen Zugriff
    \item wird von Java wie ein Objekt behandelt $\rightarrow$ liegt auf dem Heap
\end{itemize}
\begin{lstlisting}
    int[] myArray = new int[5];
    int[] mySecondArray = {1,2,3,4,5};
\end{lstlisting}
\end{frame}
\subsection{Main Method}
\begin{frame}[fragile]
\frametitle{Main Method}
Die Main Method hat als Parameter ein \textit{String} Array.\footnote{Achtung! Das args Array ist NICHT die Standardeingabe} Der Inhalt des Arrays muss beim Aufruf des Programmes mit übergeben werden.
\begin{lstlisting}
    public static void main (String[] args){
        String msg = args[0];
    }
\end{lstlisting}
Arrays haben Indices. Von 0 bis n-1.\footnote{wenn n die Länge des Arrays ist}
Wie können wir nun durch ein Array iterieren?
\begin{lstlisting}
    public static void main (String[] args){
        for(int i = 0; i < args.length; i++){
            System.out.println(args[i]);
        }
    }
\end{lstlisting}
\end{frame}

\begin{frame}[fragile]
\frametitle{Beispiel Aufgabe}
Schreiben Sie ein Programm, welches einen Wert n entgegen nimmt und eine Treppe mit Sternen ausgibt. In der M-ten Iteration sollen genau M Sterne in der Zeile ausgegeben werden. Es soll n Iterationen geben. Für n = 4:
\begin{lstlisting}
    *
    **
    ***
    ****
\end{lstlisting}
\end{frame}
\subsection{Standarteingabe}
\begin{frame}[fragile]
\frametitle{Standardeingabe}
Wir können auch die Standardeingabe benutzen, um unser Programm mit Daten aufzurufen.
Dafür benötigen wir Scanner:
\begin{lstlisting}
 import java.util.Scanner;
 public class Wuerfel {
    public static void main(String[] args) {
        Scanner stdin = new Scanner(System.in);
        while(stdin.hasNext()){
            int number = Integer.parseInt(stdin.nextLine());
            while(number > 0){
                for(int i = 0; i < number; i++){
                    System.out.print("*");
                }
                number--;
                System.out.println();
\end{lstlisting}    
\end{frame}
\begin{frame}[fragile]
\frametitle{Scanner \& Dateien}
Wir können Scanner aber auch auf spezifische Dateien setzen. Dafür brauchen wir jedoch den Dateipfad.
\begin{lstlisting}
    Scanner scanner = new Scanner(dateipfad);
\end{lstlisting}
Achtung! Wenn an diesem Pfad keine Datei liegt, wird unser Programm abstürzen. Wir sollten deshalb einen \textit{try-catch} Block benutzen:
\begin{lstlisting}
    try{
        Scanner scanner = new Scanner("/NKT/wichtigeDatei.txt");
    } catch(Exception e){
        e.printStacktrace();
    }
\end{lstlisting}
\end{frame}
\section{Mini-Game}
\begin{frame}[fragile]
\frametitle{Mini-Game}
Entwerfen Sie eine kleines Mini-Game. Sie sollen über die Standardeingabe mit dem Spiel interagieren können. 
\begin{enumerate}
    \item Ihr Spieler hat beispielsweise ein Inventar mit 3*9 Slots. Dieses wird ausgeben wenn Sie bspw \textit{inventory} in die Konsole schreiben
    \item Implementieren Sie, dass die Gegenstände in das Inventar schreiben können (\textit{add})
    \item Implementieren sie eine Methode \textit{find} <Item>, die ausgibt, an welchem Platz ein Item im Inventar liegt. (Achten Sie darauf, dass nur ein Index gefunden werden kann, wenn der Gegenstand auch wirklich im Inventar liegt)
\end{enumerate}  
\end{frame}
\begin{frame}[fragile]
\frametitle{Mini Game}
Beispielaufrufe:
\begin{lstlisting}
    $ inventory
    [Schwert][][][][][Helm][][][]
    [][][Bogen][][][][][][Trank]
    [][Holz][][][][][Relikt][][]
    $ add Stein 2
    Stein wurde an index 2 hinzugefuegt!
    $ inventory
    [Schwert][][Stein][][][Helm][][][]
    [][][Bogen][][][][][][Trank]
    [][Holz][][][][][Relikt][][]
    $ find Helm
    Ein Helm befindet sich an index 5!
\end{lstlisting}    
\end{frame}
\begin{frame}[fragile]
\frametitle{Beispielaufgabe}
Für Schnelle: 
Werden Sie kreativ und überlegen Sie sich Erweiterungen. Ein paar Denkanstöße:
\begin{itemize}
    \item der Spieler könnte ein Level haben
    \item Stats (HP, Stamina, etc)
    \item lasse den Spieler Heiltränke nutzen
    \item der Spieler kann erkunden gehen und gegen Gegner kämpfen
\end{itemize}
\end{frame}
\begin{frame}[fragile]
\frametitle{Beispielaufrufe}
\begin{lstlisting}
    $ status
    Level: 1
    HP: 100
    Stamina: 50
    $ explore
    Du findest einen Heiltrank!
    $ add Heiltrank 10
    Heiltrank wurde an Slot 10 hinzugefuegt!
    $ use Heiltrank
    Du hast einen Heiltrank benutzt und 20 HP wiederhergestellt!
    $ attack
    Du greifst den Goblin an und verursachst 15 Schaden!
    Der Goblin greift dich an und verursacht 10 Schaden!
\end{lstlisting}  
\end{frame}